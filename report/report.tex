\documentclass{article}

\usepackage[margin=1.5in]{geometry}
\usepackage{titling}
\usepackage{multicol}

\usepackage{textcomp}
\usepackage{listings}
\usepackage[usenames,dvipsnames,svgnames,table]{xcolor}

\usepackage[pdftitle={A Rustic JavaScript Interpreter}]{hyperref}

\hypersetup{
    citecolor=red,
    colorlinks=true,
    linkcolor=blue,
    filecolor=magenta,
    urlcolor=cyan,
}

% Define syntax for JavaScript
\lstdefinelanguage{JavaScript}{
  keywords={typeof, new, true, false, catch, function, return, null, catch, switch, var, if, in, while, do, else, case, break},
  keywordstyle=\color{blue}\bfseries,
  ndkeywords={class, export, boolean, throw, implements, import, this},
  ndkeywordstyle=\color{darkgray}\bfseries,
  identifierstyle=\color{black},
  sensitive=false,
  comment=[l]{//},
  morecomment=[s]{/*}{*/},
  commentstyle=\color{purple}\ttfamily,
  stringstyle=\color{red}\ttfamily,
  morestring=[b]',
  morestring=[b]"
}

\definecolor{lightergray}{gray}{0.95}

% Use JavaScript syntax for code segments
\lstset{
   language=JavaScript,
   backgroundcolor=\color{lightergray},
   extendedchars=true,
   basicstyle=\footnotesize\ttfamily,
   showstringspaces=false,
   showspaces=false,
   numbers=none,
   numberstyle=\footnotesize,
   numbersep=9pt,
   tabsize=2,
   breaklines=true,
   showtabs=false,
   captionpos=b
}

\setlength{\droptitle}{-7em}

\title{js.rs -- A Rustic JavaScript Interpreter \\
  {\large CIS Department Senior Design 2015-2016 \textsuperscript{1}}
}
\date{April 25, 2016}
\author{
  Terry Sun\\ terrysun@seas.upenn.edu \and
  Sam Rossi\\ samrossi@seas.upenn.edu
}

\setlength\parindent{0pt}

\usepackage[hang]{footmisc}
\setlength{\footnotemargin}{0pt}

\begin{document}

\maketitle

\begin{multicols}{2}

\section*{Abstract}

\footnotetext[1]{Advised by Dr. Zdancewic (stevez@cis.upenn.edu)}

JavaScript is an incredibly widespread language, running on virtually every
modern computer and browser, and interpreters such as NodeJS allow JavaScript to
be used as a server-side language. Unfortunately, modern implementations of
JavaScript engines are typically written in C/C++, languages reliant on manual
memory management. This results in countless memory leaks, bugs, and security
vulnerabilities related to memory mis-management. \newline

Js.rs is a prototype server-side JavaScript interpreter in Rust, a new systems
programming language for building programs with strong memory safety guarantees
and speeds comparable to C++. Our interpreter runs code either from source files
or an interactive REPL (read-evaluate-print-loop), similar to the functionality
of existing server-side JavaScript interpreters. We intend to demonstrate the
viability of using Rust to implement JavaScript by implementing a core subset of
language features. To that end, we’ve tested our coverage using Google’s Sputnik
test suite, an ECMAScript 5 conformance test suite.

\section{Introduction}

\subsection{Project}

TODO: define scope of project

\subsection{Motivation}

TODO

\section{Background}

\subsection{Rust}

Rust is a programming language spearheaded by Mozilla. It is a
general-purpose programming language emphasizing memory safety and speed
concerns. Rust's initial stable release (version 1.0) was released in March
2015. \newline

Rust guarantees memory safety in a unique way compared to other commonly used
programming languages. The compiler statically analyzes the source code,
tracking the ``lifetime'' of any heap-allocated data. When all existant pointers
to a piece of data has gone out of scope (e.g. at the end of a function), then
the compiler determines that the data is no longer alive and the allocated space
can be freed. (TODO: this simplifies a bit, do we want to go into more detail?)
This memory management system prevents many classic memory errors by disallowing
the programmer from accessing uninitialized or already-freed memory. \newline

Compare this system to other memory-safe languages, which rely on garbage
collection to find and free memory which is no longer relevant to the running
program. Garbage collection incurs significant runtime performance costs.
\newline

Rust code compiles down to a native executable. The Rust compiler uses an
LLVM-backend to emit assembly, taking advantage of LLVM's extensive optimization
options. In addition, there is no runtime associated with executing Rust
binaries; Rust is not run by a virtual machine (e.g. Java), nor does it use
garbage collection during execution. \newline

Rust also provides high-level programming language features, such as an type
system based on interfaces (known as Traits) and generics, funtional programming
and closures, and a set of concurrency primitives. This makes it a very
attractive languages for people who are concerned about the performance of their
program, but wish to use something more ergonomic than C or C++. The Rust
standard library is written with both performance and ergonomics in mind.

\subsection{JavaScript}

JavaScript is an extremely widely used programming language. It is one of the
core languages used on the Internet; it is included on the majority of the
websites across the Internet, and is supported by all major web browsers. In
addition, it has many desktop applications, e.g., used as a scripting
language, or embedded in PDFs. \newline

JavaScript is a high-level, imperative programming language. It is typically
interpreted, and is loosely typed; any variable can be passed into any function
or operator. The type system is composed of seven primitives types (Number,
Boolean, \texttt{null}, \texttt{undefined}, String, Object, Symbol). Custom
types can be created by inheriting from Object, using a prototype-based
inheritance system. JavaScript has many function features, such as first class
functions, anonymous functions, and closures. \newline

JavaScript was first developed at Netscape by Brendan Eich 1995 for inclusion
with their internet browser. Today, JavaScript is defined by the ECMAScript
specification, first released in 1997 by the European Computer Manufacturer
Association (ECMA) in response to the emergence of a few forks of the JavaScript
implementations. The current version is ECMAScript 6, released in June 2015;
ECMAScript 7 is still under development.\newline

JavaScript has been deployed as a server-side language since late 1995. It has
gained popularity as a server development language since Node.js was released in
2006.\newline

TODO: Mozilla Developer Network's official JavaScript documentation\cite{mozdocs}.
There is also a conformance test suite named Sputnik, released by Google in
2009. Sputnik targets the ECMAScript 5 standard, but includes only those
features which were present in ECMAScript 3. This arrangement was chosen due to
the presence of numerous ambiguities in the third specification, which were
resolved in ECMAScript 5.

\section{Design}

\subsection{Parser}

The first part of the interpreter is the parser,
takes in a string of JavaScript and generates an Abstract Syntax Tree (AST).
An AST contains the structure and the content of a program, but not the specific
syntactic components such as whitespacing. \newline

We wrote our parser using a parser generator, a common pattern in building
parsers. This consists of defining the grammar of the language (i.e. the valid
tokens of the language and the valid sequences of those tokens which have
semantic meaning in the language); the parser generator takes this grammar
specification as input and generates code to parse the grammar in the target
language (in this case, Rust). Parser generators tend to be a bit slower in
practice than writing an equivalent parser manually, but using a parser generate
greatly increases the rate of development due to the ease of use. We opted to
use a parser generator rather than writing a custom parser, as we wanted to
focus on language implementation rather than performance. \newline

The two most widely used parser generators are YACC and Bison, which are
implemented in C. Neither of these would be suitable for our project, as we
intended to use only Rust libraries in our interpreter to maintain the safety
guarantees. After some research, we decided to use LALRPOP, a pure Rust LR(1)
parser generator\cite{lalrpop}.

\subsection{Garbage Collector}

JavaScript is a garbage collected language, as it relies on a runtime system to
manage the memory used by a running program. This runtime must allocate memory
when requested by the language, and free memory when it is determined that
allocated data is no longer accessable by the running program. We used a garbage
collection library, French Press. French Press was being developed concurrency
by David Mally, a UPenn Masters student, for his Masters thesis. Js.rs is
tightly coupled with French Press, sharing a common library defining a shared
representation of the JavaScript type system.

\subsubsection*{Type System}

In JavaScript, a value can either be located on the stack or on the heap.
Primitive values (e.g. numbers, boolean) are located on the stack, whereas as
all references (such as objects or anything created from a constructor) are
located on the heap. To model this properly, our type system used a pair of
types for each value. \texttt{JsVar} contains the type of the value, as well as
any stack-related value (such as the Rust number or boolean containing the
value itself); additionally, each \texttt{JsVar} can optionally be paired with a
\texttt{JsPtrEnum}, which contains a reference to any heap allocated data that
the value may require. Each \texttt{JsVar} also contains an optional unique
binding, which acts as a lookup identifier for use with heap datatype provided
by French Press \newline

For example, for the object \texttt{\{ x: 3.5, y: "foo" \}}, the \texttt{JsVar}
would indicate that the value is of type ``object". The \texttt{JsPtrEnum} for
the object would contain a \texttt{HashMap} with the keys \texttt{"x"} and
\texttt{"y"}. The value for \texttt{"x"} would be a \texttt{JsVar} of type
``number" and with floating point value \texttt{3.5}, and the value for
\texttt{"y"} would be a \texttt{JsVar} with the type ``string" and a unique
binding which can be used to look up the heap data.

\subsection{Runtime}

The ``runtime'' is the most substantial portion of the interpreter, as it is
what accepts the AST as input and evaluates the language semantics represented
by each node. TODO

\subsubsection*{Functions, Scoping, and Closures}

One of the most complex parts of implementing the runtime was ensuring that it
exhibited the correct behavior with regards to functions and scoping. For simple
cases, French Press handled the correct scoping for local variables. However,
in the case of closures, the standard scoping rules would not suffice. The
following code sample demonstrates such a case:

\vspace{3mm}

\begin{lstlisting}
function f() {
    var x = 0;
    return function() {
        return x++;
    };
}

var g = f();
console.log(g());
console.log(g());

\end{lstlisting}

\vspace{3mm}

Normally, when a function returns, all of its local variables are no longer in
scope, so the garbage collector can deallocate the memory associated them.
However, in the above example, when \texttt{f} is called, it returns a new
function which has access to \texttt{x}. This means that \texttt{x} cannot
be garbage collected at least until after \texttt{g} is no longer in scope.
In order to correctly execute code cases like this, we had to add functionality
for our code to detect when a function is a closure and inform the garbage
collector of the special status of that function's scope.

\section{Results}

We implemented a significant subset of features which provide more than enough
coverage to write interesting and useful programs.

\subsection{Language Features}

We successfully implemented parsing and evaluation of the following types of
expressions and statements:

\subsubsection*{Assignment statements}

There are two ways to assign values to variable bindings in JavaScript. A
declaration (\texttt{var x = y}) will create a variable in the current scope and
assign the value $y$ to it. An assignment (\texttt{x = y}) will modify the
variable $x$ in the current scope or any parent scope if it already exists;
otherwise, it will be allocated in the global program scope and set to the value
$y$. Additionally, a variable can be declared but not set (\texttt{var x}),
which sets it to the default value \texttt{undefined}.

\subsubsection*{Literal expressions}

Literal expressions evaluate to the value they represent, and are defined in
JavaScript for each of the primitive types. For \texttt{null},
\texttt{undefined}, boolean, and numeric literals, we simply create a variable
containing the appropriate value. For more complex literals which must be
allocated on the heap (Strings, Objects, and Arrays), we first evaluate the
expression by constructing the appropriate type. For Strings, this includes
parsing escaped values and translating escaped Unicode values to their
corresponding Unicode characters. For Objects and Arrays, each sub-component in
theliteral must be iteratively parsed, and the value constructed. Then, we
allocate heap space and store the value.

\begin{itemize}
  \item Boolean: \texttt{true}, \texttt{false}
  \item Numeric: e.g. \texttt{-4}, \texttt{7.17}, \texttt{NaN}
  \item \texttt{null}
  \item \texttt{undefined}
  \item String: e.g. \texttt{"abc\textbackslash u1234"},
    \texttt{\textquotesingle foo bar baz\textquotesingle}
  \item Object: e.g. \texttt{\{ x: 3, y: \{ z: "hello" \} \} }
  \item Array: e.g. \texttt{[1, "hello", x]}
\end{itemize}

\subsubsection*{Operator expressions}

JavaScript provides a set of binary (\texttt{a + b}) and unary operators
(\texttt{++a}) to define expressions based on one or more variables. JavaScript
does not enforce type bounds at all, so Js.rs must perform type coercion when an
expression contains values of two different types. For example, in a logical
and statement, both expressions are first coerced to an boolean value before the
logical and is applied.\newline

This may involve operator overloading, as in the case of \texttt{+}: in most cases, the
interpreter must coerce both arguments to Number before evaluating the
expression. However, if either argument is a String, then both
arguments should instead be coerced to Strings and the operation results in the
concatentation of those two Strings. \newline

Additionally, all binary operators can be used as assignment operators (e.g.,
\texttt{a += b}. This application in Js.rs is handled by the parser, which will
produce an AST node identical to the \texttt{a = a + b} case.

\begin{itemize}
  \item Incrementation: \texttt{++}, \texttt{--}
  \item Bitwise logical: \texttt{\&}, \texttt{|}, \texttt{~}, \texttt{\^}
  \item \texttt{instanceof}
  \item \texttt{typeof}
  \item Arithmetic: \texttt{+}, \texttt{-}, \texttt{*}, \texttt{/}, \texttt{\%}
  \item Boolean logical: \texttt{\&\&}, \texttt{||}, \texttt{!}
  \item Shifts: \texttt{>>>}, \texttt{>>}, \texttt{<<}
  \item Inequalties: \texttt{>}, \texttt{>=}, \texttt{<}, \texttt{<=}
  \item Equalities: \texttt{==}, \texttt{!=}, \texttt{===}, \texttt{!==}
  \item Assignment: \texttt{=}, \texttt{+=}, \texttt{\&\&=}, ...
\end{itemize}

\subsubsection*{Function-related expressions}

Functions are stored into the current scope as a special meta-level type which
contains the function name and the AST node for the function body. When a
function is called, the interpreter allocates a new scope as a child of the
current scope, and will then execute the semantics stored in the function's
AST node. \newline

As discussed above, anonymous functions may contain references to its parent
scope, creating a closure environment.

\begin{itemize}
  \item Named function definition: e.g. \texttt{function f(x) \{ return x + 1; \} }
  \item Anonymous function definition: e.g. \texttt{function(x, y) \{ return x + y; \} }
  \item Function calls: e.g. \texttt{foo(a, b)}
\end{itemize}

\subsubsection*{Object-related expressions}

Js.rs supports simple Object functionality as a key-value store. Js.rs does not
support the Prototype inheritance model.

\begin{itemize}
  \item Instance variable access: e.g. \texttt{foo.bar}
  \item Key-indexed access: e.g. \texttt{foo[bar]}
  \item Constructors: e.g. \texttt{new foobar(x, y)}
\end{itemize}

\subsubsection*{Control-flow statements}

Many JavaScript constructions change the execution flow of the program.
Typically, the interpretere will linearly execute the AST nodes as returned by
the parser. However, the interpreter may traverse into previously encountered
nodes (e.g., a function call), repeat the current node until a condition is met
(e.g., for and while loops), or skip certain AST nodes (e.g., an if statement).

\begin{itemize}
  \item \texttt{if}/\texttt{else if}/\texttt{else}
  \item \texttt{while}, \texttt{for} loops
  \item \texttt{break}, \texttt{continue}
  \item \texttt{return}
  \item \texttt{throw}, \texttt{try}/\texttt{catch}/\texttt{finally}
\end{itemize}

\subsection*{Standard library}

We implemented a small standard library for our interpreter based on some of the
most widely-used features provided by the official JavaScript standard library.

\subsubsection*{Printing}

JavaScript interpreters typically provide a global \texttt{console} object,
through which programs can accept input and provide output to the developer
console. We implemented an abbreviated \texttt{console} containing only the
\texttt{log} function, which coerces its first argument to a string value and
outputs it to the screen.

\subsubsection*{Prototypes}

JavaScript packages a number of built-in prototypes, such as \texttt{String},
\texttt{Array}, \texttt{Object}, and \texttt{Function}. A value of a given
prototype can be created with the \texttt{new} keyword, e.g., \texttt{new
Object()}. Js.rs implements many of these prototypes as simply wrapper types
around the primitive \texttt{String}, \texttt{Number}, and \texttt{Boolean}
types. We implemented the \texttt{new} keyword by executing a call to the
appropriate constructor. \newline

Additionally, Js.rs packages a basic \texttt{Array} prototype, with support for
constructing Array objects from literals, \texttt{push}, accessing elements by
index, and re-sizing by setting the length property.

\subsection{Specification Coverage}

To test how complete our coverage of the JavaScript standard was, we built a
framework to run the Google Sputnik test suite\cite{sputnik} on our interpreter.
Sputnik targets the ECMAScript 5 standard, but includes only those features
which were present in ECMAScript 3. This arrangement was chosen due to the
presence of numerous ambiguities in the third specification, which were resolved
in ECMAScript 5. We used two different metrics to analyze the coverage of our
interpreter.

\subsubsection*{Category-based coverage}

Sputnik defines several categories of tests, each with various depths of
subcategories (for example, the ``Expressions" category contains, among others,
a ``Postfix Expressions" subcategory, which in turn contains the subcategories
``Postfix Increment Operator" and ``Postfix Decrement Operator"). Overall, there
are 111 leaf categories (i.e. categories which do not contain other
categories). \newline

We considered the number of leaf categories in which we had passed some of the
tests. Of the 111 categories, we had coverage in 73 of the categories, or
65.8\%. This indicates that we covered a sizable portion of the languages
features.

\subsubsection*{Raw coverage}

Sputnik provides a total of 2427 distinct tests. In the end, Js.rs passes 18.2\%
of those tests. Despite having implemented a large subset of the commonly used
JavaScript features, there were some features that we did not have time to
implement. Some of these included:

\begin{itemize}
  \item The global \texttt{Arguments} object
  \item The conditional operator (\texttt{ ? : })
  \item The \texttt{void} operator
  \item The \texttt{delete} operator
  \item The \texttt{in} operator
  \item The \texttt{with} statement
\end{itemize}

\section{Ethics and Privacy}

Js.rs is developed as a free and open source software project, hosted on GitHub.
Potential users do not have to blindly trust that the interpreter is not
malicious, but can audit it for themselves. As running this project on a
JavaScript program would require revealing the full source to the interpreter,
it is important that Js.rs does not contain any information exfiltration systems
or other backdoors.

\section{Discussion}

\subsubsection*{LALRPOP errata}

While LALRPOP generally worked quite well for our purposes, there were a few
issues we ran into during the development process.

\subsubsection*{Compilation speed}

While parser generators greatly facilitate development, they tend to be less
performant than custom parsers specifically written for the source language.
Although we were not heavily concerned with the code execution speed of our
interpreter, we were fairly inconvenienced by how long it would take LALRPOP to
generate our parser. As LALRPOP only had a single author who was quite busy with
other things, we understood that we would not be able to expect as good
performance as more mainstream parser generators. Given more time to improve
js.rs, we likely would have implemented a custom parser to alleviate the issue.
That being said, several non-optimization related updates to LALRPOP released
during the course of our project greatly improved the development process,
including the addition of human-readable error messages for shift-reduce and
reduce-reduce conflicts.

\subsubsection*{Lexing}

Unlike more mainstream parser generators, LALRPOP does not provide any built-in
way to use a custom lexer. By default, LALRPOP will tokenize the input by
splitting on any whitespace (and throwing out the whitespace itself). While this
would work well in many scenarios, JavaScript interpreters typically allow
newlines to be used in place of trailing semicolons. Additionally, correctly
parsing single-line comments logically requires the detection of newline
characters. In order to replicate this behavior as closely as possible with
LALRPOP, we resorted to preprocessing the code run through js.rs before parsing
it. Like the issue with compilation speed, we likely would have solved this by
writing a custom parser if we had had enough time.

\subsection*{Multi-package architecture}

One of the choices we made early on in the development of js.rs was to split the
interpreter into multiple packages. Although this originally was intended to
separate the code of French Press from the rest of the interpreter, soon decided
to split out other parts of the interpreter as well. We ended up with four
different packages, namely a \"common\" package with code needed by all other
packages, French Press, the parser, and the runtime. Although this logical
separation made sense in the abstract, it quickly became cumbersome to ensure
that each crate's dependencies were in sync (e.g. that the parser and the
runtime both used the same version of the common package). Moreover, due to the
slow speed at which the parser compiled, changes to the common package became
rather undesirable, as each change would require the parser to recompile as
well. In retrospect, using separate modules within a single package rather than
multiple packages would have significantly increased our development efficiency.

\section{Conclusion}

TODO

\end{multicols}

\pagebreak

\begin{thebibliography}{9}
  \bibitem{rust-lang}
  Official Rust Language website
  \href{https://www.rust-lang.org/}{https://www.rust-lang.org/}

  \bibitem{lalrpop}
  LALRPOP, LR(1) parser generator for Rust
  \href{https://github.com/nikomatsakis/lalrpop}{https://github.com/nikomatsakis/lalrpop}

  \bibitem{mozdocs}
  Mozilla Developer Network JavaScript Documentation
  \href{https://developer.mozilla.org/en-US/docs/Web/JavaScript}{https://developer.mozilla.org/en-US/docs/Web/JavaScript}

  \bibitem{sputnik}
  ECMAScript conformance test suite
  \href{https://code.google.com/archive/p/sputniktests/}{https://code.google.com/archive/p/sputniktests/}
\end{thebibliography}

\end{document}
