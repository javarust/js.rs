\documentclass{article}

\usepackage[margin=1.5in]{geometry}
\usepackage{titling}

\usepackage{listings}
\usepackage[usenames,dvipsnames,svgnames,table]{xcolor}

% Define syntax for JavaScript
\lstdefinelanguage{JavaScript}{
  keywords={typeof, new, true, false, catch, function, return, null, catch, switch, var, if, in, while, do, else, case, break},
  keywordstyle=\color{blue}\bfseries,
  ndkeywords={class, export, boolean, throw, implements, import, this},
  ndkeywordstyle=\color{darkgray}\bfseries,
  identifierstyle=\color{black},
  sensitive=false,
  comment=[l]{//},
  morecomment=[s]{/*}{*/},
  commentstyle=\color{purple}\ttfamily,
  stringstyle=\color{red}\ttfamily,
  morestring=[b]',
  morestring=[b]"
}

\definecolor{lightergray}{gray}{0.95}

% Use JavaScript syntax for code segments
\lstset{
   language=JavaScript,
   backgroundcolor=\color{lightergray},
   extendedchars=true,
   basicstyle=\footnotesize\ttfamily,
   showstringspaces=false,
   showspaces=false,
   numbers=none,
   numberstyle=\footnotesize,
   numbersep=9pt,
   tabsize=2,
   breaklines=true,
   showtabs=false,
   captionpos=b
}

\setlength{\droptitle}{-7em}

\title{js.rs -- A Rustic JavaScript Interpreter}
\date{April 25, 2016}
\author{Terry Sun, Sam Rossi}

\setlength\parindent{0pt}

\begin{document}

\maketitle

\section*{Introduction}

JavaScript is an incredibly widespread language, running on virtually every
modern computer and browser, and interpreters such as NodeJS allow JavaScript to
be used as a server-side language. Unfortunately, modern implementations of
JavaScript engines are typically written in C/C++, languages reliant on manual
memory management. This results in countless memory leaks, bugs, and security
vulnerabilities related to memory mismanagement. \newline

We’ve chosen to build a prototype server-side JavaScript interpreter in Rust, a
new systems programming language for building programs with strong memory safety
guarantees and speeds comparable to C++. Our interpreter runs code either from
source files or an interactive REPL (read-evaluate-print-loop), similar to the
functionality of existing server-side JavaScript interpreters. We intend to
demonstrate the viability of using Rust to implement JavaScript by implementing
a core subset of language features. To that end, we’ve tested our coverage using
Google’s Sputnik test suite, an ECMAScript 5 conformance test suite.

\subsection*{The Rust Language}

According to the official website, Rust is "systems programming language that
runs blazingly fast, prevents segfaults, and guarantees thread
safety"\footnote{https://www.rust-lang.org/}. Designed to be safer than
traditional systems programming languages, Rust's creator described Rust as
designed for "frustrated C++
developers"\footnote{http://www.infoq.com/news/2012/08/Interview-Rust}. To that
end, Rust guarantees memory safety in a relatively unique way compared to most
mainstream programming languages; while traditional memory-safe languages use
garbage collection, which incurs significant runtime costs, Rust using a
combination of compile-time techniques including a nuanced type system, tracking
of pointer lifetimes, and several other cutting-edge techniques developed by
by programming languages researchers over the years.

\section*{Approach}

\subsection*{Parser}

The first part of the interpreter that we implemented was the parser, which
takes in a string of JavaScript and generates an "Abstract Syntax Tree", or AST.
An AST contains the structure and the content of a program, but not the specific
syntactic components such as whitespacing. \newline

Traditionally, the canonical way of writing a parser is to use a "parser
generator". Implementing a parser with a parser generator consists of defining
the grammar of the language (i.e. the valid tokens of the language and what
sequences of tokens are valid); the parser generator then generate code to parse
the grammar, which is then used as a library. Although parser generators tend to
be a bit slower in practice than writing an equivalent parser manually, using
a parser generate greatly increases the rate of development due to the ease of
use. Because of the scope and time limitations of our project, we opted to use a
parser generator rather than writing a custom parser.

The two most widely used parser generators are YACC and Bison, which are
implemented in C. Neither of these would be suitable for our project, as we
intended to use only Rust libraries in our interpreter to maximize the safety
guarantees. After some research, we decided to use LALRPOP, a pure Rust LR(1)
parser generator\footnote{https://github.com/nikomatsakis/lalrpop}.

\subsection*{Garbage Collector}

Like other interpreted languages, JavaScript relies on garbage collection to
manage the memory used by a running program. Although it would have been
possible for us to write our own garbage collector, a master's student
with the same advisor as us was working on implementing a JavaScript garbage
collection library in Rust for his thesis. Integrating with the garbage
collection library seemed mutually beneficial, so we opted not to write our own
garbage collector in favor of using French Press.

\subsection*{Runtime}

In terms of time and amount of code, the most signficant part of our work was
the actual evaluation of the JavaScript code, which is generally referred to as
the "runtime system", or juse the "runtime". In order to effectively follow the
proper semantics of JavaScript code, we made heavy use of the Mozilla Developer
Network's official JavaScript
documentation\footnote{https://developer.mozilla.org/en-US/docs/Web/JavaScript}.
This greatly reduced the amount of effort we had to spend on determining things
like the proper precendence of operators and the correct implementation of
implicit type coercions.

\subsubsection*{Functions, Scoping, and Closures}

One of the most difficult parts of implementing the runtime was ensuring that it
exhibited the correct behavior with regards to functions and scoping. For simple
cases, French Press handled the correct scoping for local variables. However,
in the case of closures, the standard scoping rules would not suffice. The
following code sample demonstrates such a case:

\vspace{3mm}

\begin{lstlisting}
function f() {
    var x = 0;
    return function() {
        return x++;
    };
}

var g = f();
console.log(g());
console.log(g());

\end{lstlisting}

\vspace{3mm}

Normally, when a function returns, all of its local variables are no longer in
scope, so the garbage collector can deallocate the memory associated them.
However, in the above example, when \texttt{f} is called, it returns a new
function which has access to \texttt{x}. This means that \texttt{x} cannot
be garbage collected at least until after \texttt{g} is no longer in scope.
In order to correctly execute code cases like this, we had to add functionality
for our code to detect when a function is a closure and inform the garbage
collector of the special status of that function's scope.

\section*{Results}

Although we did not have time to implement 100\% of the JavaScript language, we
implemented a significant subset of features which provide more than enough to
write interesting and useful programs.

\subsection*{Language Features}

We successfully implemented parsing and evaluation of the following types of
expressions and statements:

\subsubsection*{Literal expressions}

\begin{itemize}
  \item Boolean: \texttt{true}, \texttt{false}
  \item Numeric: e.g. \texttt{-4}, \texttt{7.17}, \texttt{NaN}
  \item String: e.g. \texttt{"abc\u1234"}, \texttt{'foo bar baz'}
  \item Object: e.g. \texttt{\{ x: 3, y: \{ z: "hello" \} \} }
  \item Array: e.g. \texttt{[1, "hello", x]}
  \item \texttt{null}
  \item \texttt{undefined}
\end{itemize}

\subsubsection*{Operator expressions}

\begin{itemize}
  \item Arithmetic: \texttt{+}, \texttt{-}, \texttt{*}, \texttt{/}, \texttt{\%}
  \item Incrementation: \texttt{++}, \texttt{--}
  \item Boolean logical: \texttt{\&\&}, \texttt{||}, \texttt{!}
  \item Bitwise logical: \texttt{\&}, \texttt{|}, \texttt{~}, \texttt{\^}
  \item Shifts: \texttt{>>>}, \texttt{>>}, \texttt{<<}
  \item Inequalties: \texttt{>}, \texttt{>=}, \texttt{<}, \texttt{<=}
  \item Equalities: \texttt{==}, \texttt{!=}, \texttt{===}, \texttt{!==}
  \item Assignment: \texttt{=}, \texttt{+=}, \texttt{\&\&=}, ...
  \item \texttt{instanceof}
  \item \texttt{typeof}
\end{itemize}

\subsubsection*{Function-related expressions}

\begin{itemize}
  \item Named function definition: e.g. \texttt{function f(x) \{ return x + 1; \} }
  \item Anonymous function definition: e.g. \texttt{function(x, y) \{ return x + y; \} }
  \item Function calls: e.g. \texttt{foo(a, b)}
\end{itemize}

\subsubsection*{Object-related expressions}

\begin{itemize}
  \item Instance variable access: e.g. \texttt{foo.bar}
  \item Key-indexed access: e.g. \texttt{foo[bar]}
  \item Constructors: e.g. \texttt{new foobar(x, y)}
\end{itemize}

\subsubsection*{Mutation statements}

\begin{itemize}
  \item Assignment: e.g. \texttt{x = y}, \texttt{foo[bar] = 23}
  \item Declarations: e.g. \texttt{var x = z}
\end{itemize}

\subsubsection*{Control-flow statements}

\begin{itemize}
  \item \texttt{if}/\texttt{else if}/\texttt{else}
  \item \texttt{while}
  \item \texttt{for}
  \item \texttt{break}
  \item \texttt{continue}
  \item \texttt{return}
  \item \texttt{try}/\texttt{catch}/\texttt{finally}
  \item \texttt{throw}
\end{itemize}

\subsection*{Standard library}

We implemented a small standard library for our interpreter based on some of the
most widely-used features the official JavaScript standard library

\subsubsection*{Printing}

Official JavaScript interpreters do console-based I/O through the global
\texttt{console} object, which has methods such as \texttt{log()} and
\texttt{error}. Because implementing the entire native \texttt{console} object
would have taken significant time and distracted us from implementing more
widely-used features, we implemented a global \texttt{log()} function which
behaves like \texttt{console.log()}.

\subsubsection*{Prototypes}

JavaScript typically provides a number of built-in prototypes, including
\texttt{String}, \texttt{Array}, \texttt{Object}, and \texttt{Function}. Because
the majority of the other features could be easily implemented natively, we
decided to build in a simple \texttt{Array} prototype to demonstrate our
implementation of prototype functionality in our interpreter.

\subsection*{Coverage}

To test how complete our coverage of the JavaScript standard was, we built a
framework to run the Google Sputnik test
suite\footnote{https://code.google.com/archive/p/sputniktests/} on our
interpreter. The test suite covers the ECMAScript 3 subset of the ECMAScript 5
standard. We used two different metrics to analyze the coverage of our
interpreter.

\subsubsection*{Category-based coverage}

Sputnik defines several categories of tests, each with various depths of subcategories
(for example, the "Expressions" category contains, among others, a "Postfix
Expressions" subcategory, which in turn contains the subcategories "Postfix
Increment Operator" and "Postfix Decrement Operator"). Overall, there are 111
leaf categories (i.e. categories which do not contain other categories). \newline

In order to determine what percentage of the JavaScript language features we had
implemented, we counted the number of leaf categories in which we had passed
some of the tests. Of the 111, we found that we had coverage in 73 of the
categories, or 65.8\% of them. This indicates that we while we had not
implemented a full-fledged JavaScript interpreter, we covered a sizable portion
of the languages features.

\subsubsection*{Raw coverage}

\section*{Ethical and Privacy Considerations}

Although our project is novel in terms of its implementation, it isn't intended
to provide any functionality that does not exist. Given that there are existing
JavaScript interpreters that differ only in that they do not provide the same
guarantees about memory leaks and segfaults, our project has essentially no
impact on privacy or ethical concerns.

\section*{Discussion}

\end{document}
